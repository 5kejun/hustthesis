\documentclass[12pt,a4paper,numbered,full]{l3doc}

\usepackage{luatexja-fontspec}
% 英文字体
\setmainfont[Ligatures={Common,TeX}]{Tex Gyre Pagella}
\setsansfont[Ligatures={Common,TeX}]{Droid Sans}
\setmonofont[Ligatures={Common,TeX}]{CMU Typewriter Text}
\defaultfontfeatures{Mapping=tex-text,Scale=MatchLowercase}
% 中文字体
\setmainjfont[BoldFont={AdobeHeitiStd-Regular},ItalicFont={AdobeKaitiStd-Regular}]{AdobeSongStd-Light}
\setsansjfont{AdobeKaitiStd-Regular}
\defaultjfontfeatures{JFM=kaiming}
\newjfontfamily\KAI{AdobeKaitiStd-Regular}
\newjfontfamily\FANGSONG{AdobeFangsongStd-Regular}

\linespread{1.1}

\usepackage[top=1.2in,bottom=1.2in,left=1.5in,right=1in]{geometry}
\pdfpagewidth=\paperwidth
\pdfpageheight=\paperheight

\hypersetup{
  unicode,
  bookmarksnumbered=true,
  bookmarksopen=true,
  bookmarksopenlevel=2,
  breaklinks=true,
  colorlinks=true,
  linkcolor=blue,
  plainpages=false,
  pdfpagelabels=true,
  pdfstartview={XYZ null null 1}
}
\EnableCrossrefs

\usepackage{indentfirst}
\setlength{\parindent}{2em}

\usepackage{color}
\usepackage[table]{xcolor}

\usepackage{titlesec}
\setcounter{tocdepth}{2}
\setcounter{secnumdepth}{3}

\usepackage{enumitem}
\setlist{noitemsep,partopsep=0pt,topsep=.8ex}
\setlist[1]{labelindent=\parindent}
\setlist[enumerate,1]{label=\arabic*.}
\setlist[enumerate,2]{label*=\arabic*}
\setlist[enumerate,3]{label=\emph{\alph*})}

\usepackage{listings}
\definecolor{lstgreen}{rgb}{0,0.6,0}
\definecolor{lstgray}{rgb}{0.5,0.5,0.5}
\definecolor{lstmauve}{rgb}{0.58,0,0.82}
\lstset{
  basicstyle=\footnotesize\ttfamily\FANGSONG,
  keywordstyle=\color{blue}\bfseries,
  commentstyle=\color{lstgreen}\itshape\KAI,
  stringstyle=\color{lstmauve},
  showspaces=false,
  showstringspaces=false,
  showtabs=false,
  numbers=left,
  numberstyle=\tiny\color{lstgray},
  frame=lines,
  rulecolor=\color{black},
  breaklines=true
}

\usepackage{metalogo}
\usepackage{notes}
\usepackage{tabularx}
\newcommand{\tabincell}[2]{\begin{tabular}{@{}#1@{}}#2\end{tabular}}

\def\contentsname{目录}
\def\tablename{表}
\def\indexname{索引}
\def\orvar{\textnormal{|}}
\IndexPrologue{
  \section{\indexname}
}

\begin{document}

\title{华科学术论文\LaTeX{}模板:hustthesis.cls 说明文档}
\author{许铖~~陈雷}
\date{2013-4-1}

\maketitle

\tableofcontents

\section{简介}

本模板为华中科技大学\textbf{非官方}学术论文模板。希望能够帮助正忙于撰写毕业论文和学术论文的华科学子。并且我们希望借助这个模板在学校内推广\LaTeX{}的使用。

对于中文用户来说,在使用\LaTeX{}处理文档时,一个非常常见的问题即是对汉字的处理。以往常见的方法是使用\verb+CJK+宏包。但这个宏包早已停止开发,存在很多已知bug,并与很多宏包不兼容。\XeTeX{}和\LuaTeX{}的出现则很好的解决了这个问题。他们都规定以UTF-8这个通用编码方式来处理文档。并且在这两种引擎下都有相应宏包来处理中文。在\XeLaTeX{}下,有\href{http://mirrors.ctan.org/help/Catalogue/entries/xecjk.html}{\texttt{xeCJK}};在\LuaLaTeX{}下,有\href{http://mirrors.ctan.org/help/Catalogue/entries/luatexja.html}{\texttt{luatex-ja}}。这两个宏包当前的开发状态都保持活跃。本模板同时支持这两种宏包来处理中文,并会根据编译所用的引擎自动选择。

本模板基于原有的\href{http://sourceforge.net/projects/hustthesis}{\texttt{HUSTPHDthesis.cls}  (2006/04/06 V2.0)}重写而成,并以LGPL v2.1为协议发布于\href{https://github.com/michael911009/HUSTThesis}{GitHub}上。

\section{使用必要条件}

\begin{enumerate}
    \item 安装最新版本的\href{http://www.tug.org/texlive/}{\texttt{Texlive}}(推荐)或\href{http://miktex.org/}{\texttt{MiKTex}}。因为未及时更新的宏包可能存在未修复的bug,请确保所有宏包都更新至最新。
    \item 安装如下中文字体\footnote{本模板所用到的英文字体\texttt{Tex Gyre Termes},\texttt{Droid Sans}和\texttt{CMU Typewriter Text}均默认安装于\texttt{Texlive}和\texttt{MiKTex}中。}:
    \begin{enumerate}[label=\emph{\alph*})]
        \item \verb+AdobeSongStd-Light+
        \item \verb+AdobeKaitiStd-Regular+
        \item \verb+AdobeHeitiStd-Regular+
        \item \verb+AdobeFangsongStd-Regular+
    \end{enumerate}
    \begin{informationnote}
    如果使用\textnormal{\LuaTeX},安装字体之后需运行命令\verb+mkluatexfontdb+生成字体索引。
    \end{informationnote}
\end{enumerate}

\section{安装}

\subsection{安装到本地}

% TODO:

\subsection{免安装使用}

如果你希望临时使用本模板,而非安装到本地供长期使用。将\verb+hustthesis+目录下的如下文件拷贝到你\TeX{}工程根目录下即可:
\begin{enumerate}
    \item \verb+hustthesis.bst+
    \item \verb+hustthesis.cls+
    \item \verb+hust-thesis-var.tex+
    \item \verb+hust-title.eps+
    \item \verb+hust-title.pdf+
\end{enumerate}

\section{基本用法}

\begin{importantnote}
本文档只能使用\textnormal{\XeLaTeX}或\textnormal{\LuaLaTeX}(推荐)编译。
\end{importantnote}

在源文件开头处选择加载本文档类型,即可使用本模板,如下所示:
\begin{verbatim}
    \documentclass{hustthesis}
\end{verbatim}

\subsection{文档类型选项}

加载本文档类型时,有两个选项提供选择。

\begin{function}{format}
    \begin{syntax}
        format = \meta{draft\orvar{}\textbf{final}}
    \end{syntax}
    提交草稿选择\verb+draft+选项,提交最终版选\verb+final+选项。其中草稿正文页包括页眉(“华中科技大学\verb+xx+学位论文”)、页眉修饰线(双线)、页脚(页码)和页脚修饰线(单线)。而最终版正文页不包括页眉、页眉修饰线和页脚修饰线,仅包含页脚(页码)。如果不指定,默认设置为\verb+final+。
\end{function}

\begin{function}{degree}
    \begin{syntax}
        degree = \meta{\textbf{none}\orvar{}fyp\orvar{}bachelor\orvar{}master\orvar{}phd}
    \end{syntax}
    指定论文种类,它将通过设置字段\verb+\HUST@zhapplyname+和\verb+\HUST@enapplyname+来影响中英文封面除的标题和正文处的页眉(如果\verb+format+设为\verb+draft+)。各个不同的选项产生的效果见表\ref{tab:optdegree}。如果不指定,默认设置为\verb+none+。
\end{function}

\begin{table}[ht]
    \centering
    \caption{\texttt{degree}选项的作用}\label{tab:optdegree}
    \begin{tabularx}{\textwidth}{|c|X|X|}
    \hline
    \textbf{选项} & \tabincell{c}{\textbf{中文标题}\\(字段\verb+\HUST+\verb+@zhapplyname+)} & \tabincell{c}{\textbf{英文标题}\\(字段\verb+\HUST+\verb+@enapplyname+)} \\
    \hline
    \verb+none+ & 学位论文 & A Thesis Submitted in Partial Fulfillment of the Requirements for the Degree \\ \hline
    \verb+fyp+ & 毕业论文 & A Thesis Submitted in Partial Fulfillment of the Requirements for Final Year Project \\ \hline
    \verb+bachelor+ & 学士学位论文 & A Thesis Submitted in Partial Fulfillment of the Requirements for the Degree of Bachelor \\ \hline
    \verb+master+ & 硕士学位论文 & A Thesis Submitted in Partial Fulfillment of the Requirements for the Degree of Master \\ \hline
    \verb+phd+ & 博士学位论文 & A Thesis Submitted in Partial Fulfillment of the Requirements for the Ph.D \\ \hline
    \end{tabularx}
\end{table}

\subsection{基本字段设置}

模板中定义一些命令用于设置论文中的字段。其中一部分命令是以\verb+\zhX{<Chinese var>}+,\verb+\enX{<English var>}+和\verb+\X{<Chinese var>}{<English var>}+的形式出现,他们分别意味着设置字段\verb+X+的中文部分,英文部分及一同设定。

\begin{function}{\zhtitle,\entitle,\title}
    \begin{syntax}
    \cs{zhtitle}\Arg{Chinese title}
    \cs{entitle}\Arg{English title}
    \cs{title}\Arg{Chinese title}\Arg{English title}
    \end{syntax}
    这一组命令用于设定论文的中英文标题。
\end{function}

\begin{function}{\zhauthor,\enauthor,\author}
    \begin{syntax}
    \cs{zhauthor}\Arg{Chinese author}
    \cs{enauthor}\Arg{English author}
    \cs{author}\Arg{Chinese author}\Arg{English author}
    \end{syntax}
    这一组命令用于设定论文的中英文作者名。
\end{function}

\begin{function}{\date}
    \begin{syntax}
    \cs{date}\Arg{Year}\Arg{Month}\Arg{Day}
    \end{syntax}
    该命令用于设定论文的答辩日期。如果不设定,则会选择当前编译日期作为答辩日期。
\end{function}

\begin{function}{\zhschoolname,\enschoolname,\schoolname}
    \begin{syntax}
    \cs{zhschoolname}\Arg{Chinese schoolname}
    \cs{enschoolname}\Arg{English schoolname}
    \cs{schoolname}\Arg{Chinese schoolname}\Arg{English schoolname}
    \end{syntax}
    这一组命令用于设定论文的中英文学校名。该字段在模板中已默认设置为\verb+\schoolname{华中科技大学}{Huazhong University of Science \& Technology}+。所以除非需更改学校名,无需使用该命令。
\end{function}

\begin{function}{\zhaddress,\enaddress,\address}
    \begin{syntax}
    \cs{zhaddress}\Arg{Chinese address}
    \cs{enaddress}\Arg{English address}
    \cs{address}\Arg{Chinese address}\Arg{English address}
    \end{syntax}
    这一组命令用于设定论文的中英文学校地址。该字段在模板中已默认设置为\verb+\address{中国,武汉,430074}{Wuhan~430074, P.~R.~China}+。所以除非需更改学校地址,无需使用该命令。
\end{function}

\begin{function}{\zhmajor,\enmajor,\major}
    \begin{syntax}
    \cs{zhmajor}\Arg{Chinese major}
    \cs{enmajor}\Arg{English major}
    \cs{major}\Arg{Chinese major}\Arg{English major}
    \end{syntax}
    这一组命令用于设定论文的中英文专业名。
\end{function}

\begin{function}{\zhsupervisor,\ensupervisor,\supervisor}
    \begin{syntax}
    \cs{zhsupervisor}\Arg{Chinese supervisor}
    \cs{ensupervisor}\Arg{English supervisor}
    \cs{supervisor}\Arg{Chinese supervisor}\Arg{English supervisor}
    \end{syntax}
    这一组命令用于设定论文的中英文指导老师名(含职称)。
\end{function}

\begin{function}{\zhasssupervisor,\enasssupervisor,\asssupervisor}
    \begin{syntax}
    \cs{zhasssupervisor}\Arg{Chinese asssupervisor}
    \cs{enasssupervisor}\Arg{English asssupervisor}
    \cs{asssupervisor}\Arg{Chinese asssupervisor}\Arg{English asssupervisor}
    \end{syntax}
    这一组命令用于设定论文的中英文副指导老师名(含职称)。该命令是可选的,如果不加以设定,封面处不会显示相应项。
\end{function}

\begin{function}{\schoolcode}
    \begin{syntax}
    \cs{schoolcode}\Arg{school code}
    \end{syntax}
    用于设置学校代码。
\end{function}

\begin{function}{\stuno}
    \begin{syntax}
    \cs{stuno}\Arg{student number}
    \end{syntax}
    用于设置学号。
\end{function}

\begin{function}{\classno}
    \begin{syntax}
    \cs{classno}\Arg{classify number}
    \end{syntax}
    用于设置分类号。
\end{function}

\begin{function}{\secretlevel}
    \begin{syntax}
    \cs{secretlevel}\Arg{secret level}
    \end{syntax}
    用于设置密级。
\end{function}

\begin{function}{\zhabstract,\enabstract,\abstract}
    \begin{syntax}
    \cs{zhabstract}\Arg{Chinese abstract}
    \cs{enabstract}\Arg{English abstract}
    \cs{abstract}\Arg{Chinese abstract}\Arg{English abstract}
    \end{syntax}
    这一组命令用于设定论文的中英文摘要。
\end{function}

\begin{function}{\zhkeywords,\enkeywords,\keywords}
    \begin{syntax}
    \cs{zhkeywords}\Arg{Chinese keywords}
    \cs{enkeywords}\Arg{English keywords}
    \cs{keywords}\Arg{Chinese keywords}\Arg{English keywords}
    \end{syntax}
    这一组命令用于设定论文的中英文关键字。
\end{function}

\subsection{其它基本命令}

下面来介绍其它基本命令。

\begin{function}{\frontmatter,\mainmatter,\backmatter}
    这一组命令用于设定论文的状态、改变样式,其具体使用见\nameref{subsec:简单示例}。\verb+\frontmatter+用在文档最开始,表明文档的前言部分(如封面,摘要,目录等)的开始。\verb+\mainmatter+表示论文正文的开始。\verb+\backmatter+表示论文正文的结束。
\end{function}

\begin{function}{\maketitle,\makecover}
    \verb+\maketitle+和\verb+\makecover+作用相同,用于生成封面和版权页面。
\end{function}

\begin{function}{\makeabstract}
    用于生成中英文摘要页面。
\end{function}

\begin{function}{\tableofcontents}
    用于生成目录。
\end{function}

\begin{function}{ack}
    \begin{syntax}
    \verb+\begin{ack}+
    \hspace{2em}\meta{content}
    \verb+\end{ack}+
    \end{syntax}
    \verb+ack+环境用于致谢页面。
\end{function}

\begin{function}{\bibliography}
    \begin{syntax}
    \cs{bibliography}\Arg{.bib file}
    \end{syntax}
    用于生成参考文献。
\end{function}

\begin{function}{appendix}
    \verb+appendix+环境用于附录环境。你可以将附录置于\verb+appendix+环境中,如:
    \begin{verbatim}
    \begin{appendix}
        <content>
    \end{appendix}
    \end{verbatim}
    或者使用\verb+\appendix+代表后文均为附录,如:
    \begin{verbatim}
    \appendix
    <content>
    \end{verbatim}
\end{function}

\begin{function}{\listoffigures,\listoftables}
    这两个命令分别用于生成图片和表格索引,可以根据要求在论文前言中使用或附录中使用。
\end{function}

\begin{function}{publications}
    \begin{syntax}
    \verb+\begin{publications}+
    \hspace{2em}\cs{item} \meta{thesis}
    \hspace{2em}\meta{$\cdots$}
    \verb+\end{publications}+
    \end{syntax}
    \verb+publications+环境用于已发表论文页面。一般用于附录中。使用上同\verb+enumerate+环境。
\end{function}

\begin{function}{\TurnOffTabFontSetting,\TurnOnTabFontSetting}
    因为模板中设定了表格的行距和字号,使得使用中无法临时自定义表格的行距和字号。故提供两个命令用于关闭和开启默认表格的行距和字号设置。比如你如果需要输出一个自己定义字号的表格,可以使用如下示例:
    \begin{verbatim}
    \begingroup
    \TurnOffTabFontSetting
    \footnotesize % 设置字号
    \begin{tabular}{...}
        <content>
    \end{tabular}
    \TurnOnTabFontSetting
    \endgroup
    \end{verbatim}
\end{function}

\subsection{简单示例}\label{subsec:简单示例}
如下为一个使用本模板的简单示例。更完整的例子请见\href{https://github.com/michael911009/HUSTThesis/blob/master/example/hustthesis-example.tex}{\texttt{hustthesis-example.tex}},其效果见\href{https://github.com/michael911009/HUSTThesis/raw/master/example/hustthesis-example.pdf}{\texttt{hustthesis-example.pdf}}。

\begin{lstlisting}[language={[LaTeX]TeX}]
\documentclass[degree=phd]{hustthesis}

\stuno{你的学号}
\schoolcode{10487}
\title{中文标题}{英文标题}
\author{作者名}{作者名拼音}
\major{专业中文}{专业英文}
\supervisor{指导老师中文}{指导老师英文}
\date{2013}{4}{1} % 答辩日期

\zhabstract{中文摘要}
\zhkeywords{中文关键字}
\enabstract{英文摘要}
\enkeywords{英文关键字}

\begin{document}

\frontmatter
\maketitle
\makeabstract
\tableofcontents
\mainmatter

%% 正文

\backmatter

\begin{ack}
%% 致谢
\end{ack}
\bibliography{参考文献.bib文件}

\appendix

\listoffigures
\listoftables

\begin{publications}
%% 发表过的论文列表
\end{publications}

%% 附录剩余部分

\end{document}
\end{lstlisting}

\section{预设宏包介绍}

本模板中预设了一些宏包,下面对其进行简单介绍。

\begin{itemize}
    \item \href{http://mirrors.ctan.org/help/Catalogue/entries/algorithm2e.html}{\texttt{algorithm2e}} 算法环境。
    \item \href{http://mirrors.ctan.org/help/Catalogue/entries/enumitem.html}{\texttt{enumitem}} 自定义列表环境的式样。
    \item \href{http://mirrors.ctan.org/help/Catalogue/entries/fancynum.html}{\texttt{fancynum}} 用于将大数每三位断开。
    \item \href{http://mirrors.ctan.org/help/Catalogue/entries/listings.html}{\texttt{listings}} 代码环境。如需更好的代码高亮可以使用\href{http://mirrors.ctan.org/help/Catalogue/entries/minted.html}{\texttt{minted}}宏包。
    \item \href{http://mirrors.ctan.org/help/Catalogue/entries/longtable.html}{\texttt{longtable}} 跨页的超长表格环境。
    \item \href{http://mirrors.ctan.org/help/Catalogue/entries/ltxtable.html}{\texttt{ltxtable}} \texttt{longtable}环境和\texttt{tabularx}环境的合并。
    \item \href{http://mirrors.ctan.org/help/Catalogue/entries/multirow.html}{\texttt{multirow}} 用于表格中合并行。
    \item \href{http://mirrors.ctan.org/help/Catalogue/entries/overpic.html}{\texttt{overpic}} 用于在图片上层叠其他内容。
    \item \href{http://mirrors.ctan.org/help/Catalogue/entries/tabularx.html}{\texttt{tabularx}} 扩展到表格环境。
    \item \href{http://mirrors.ctan.org/help/Catalogue/entries/xypic.html}{\texttt{xy-pic}} 用于绘制简单图形。如需更高级功能可以使用\href{http://mirrors.ctan.org/help/Catalogue/entries/pgf.html}{\texttt{tikz}}宏包。
    \item \href{http://mirrors.ctan.org/help/Catalogue/entries/zhnumber.html}{\texttt{zhnumber}} 用于生成中文数字。
\end{itemize}

\section{高级设置}

\subsection{切换字体}

模板正文字体为宋体(AdobeSongStd-Light),同时我们提供如下命令切换中文字体:

\begin{function}{\HEI,\hei}
    \begin{syntax}
    \{\cs{HEI} \meta{content}\}
    \cs{hei}\Arg{content}
    \end{syntax}
    切换字体为黑体(\verb+AdobeHeitiStd-Regular+)。
\end{function}

\begin{function}{\KAI,\kai}
    \begin{syntax}
    \{\cs{KAI} \meta{content}\}
    \cs{kai}\Arg{content}
    \end{syntax}
    切换字体为楷体(\verb+AdobeKaitiStd-Regular+)。
\end{function}

\begin{function}{\FANGSONG,\fangsong}
    \begin{syntax}
    \{\cs{FANGSONG} \meta{content}\}
    \cs{fangsong}\Arg{content}
    \end{syntax}
    切换字体为仿宋(\verb+AdobeFangsongStd-Regular+)。
\end{function}

如果需要加载其他字体,请参阅宏包\href{http://mirrors.ctan.org/help/Catalogue/entries/fontspec.html}{\texttt{fontspec}},宏包\href{http://mirrors.ctan.org/help/Catalogue/entries/xecjk.html}{\texttt{xeCJK}}(对于\XeLaTeX{})和宏包\href{http://mirrors.ctan.org/help/Catalogue/entries/luatexja.html}{\texttt{luatex-ja}}(对于\LuaLaTeX{})的文档。

\subsection{内部字段设置}

本模板定义了很多内部字段,其实现都位于\verb+hust-thesis-var.tex+中。其具体内容如下:
\lstinputlisting[language={[LaTeX]TeX},firstline=35]{../hustthesis/hust-thesis-var.tex}

通过更改这些字段,可以对论文格式进行自定义。

\begingroup
\hypersetup{bookmarksopenlevel=0}
\PrintIndex
\endgroup

\end{document}